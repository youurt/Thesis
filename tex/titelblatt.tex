% *******************************************************************
% In dieser Datei sollten eigentlich keine Veränderungen
% notwendig sein. Alle Einstellungen erfolgen in docinfo.tex und
% der thesis.tex.
% *******************************************************************

\thispagestyle{empty}

% Fakultäten der HS-Siegen
% *******************************************************************
\ifthenelse{\equal{\hsmafakultaet}{I}}%
{\newcommand{\hsmafakultaetlangde}{Fakultät IV - Institut für Wissensbasierte
    Systeme und Wissensmanagement}%
  \newcommand{\hsmafakultaetlangen}{}}{}


% Studiengänge der HS-Siegen
% *******************************************************************
\ifthenelse{\equal{\hsmastudiengang}{IB}}%
{\newcommand{\hsmastudienganglangde}{Wirtschaftsinformatik}%
  \newcommand{\hsmastudienganglangen}{Business Informatics}%
  \newcommand{\hsmatypde}{Master-Thesis}%
  \newcommand{\hsmatypen}{Master Thesis}%
  \newcommand{\hsmagrad}{\hsmamaster}}{}

% Abschlüsse
% *******************************************************************
\newcommand{\hsmamaster}{Master of Science (M.Sc.)}
\newcommand{\hsmamastera}{Master of Arts (M.A.)}
\newcommand{\hsmamasterba}{Master of Business Administration (MBA)}

\newcommand{\hsmakoerperschaftde}{Universität Siegen}
\newcommand{\hsmakoerperschaften}{University of Siegen}

\newcommand{\hsmaautorbib}{\hsmaautornname, \hsmaautorvname} % Autor Nachname, Vorname
\newcommand{\hsmaautor}{\hsmaautorvname \ \hsmaautornname} % Autor Vorname Nachname

\ifthenelse{\equal{\hsmasprache}{de}}%
{\newcommand{\hsmatyp}{\hsmatypde}%
  \newcommand{\hsmathesistype}{zur Erlangung des akademischen Grades \hsmagrad}%
  \newcommand{\hsmakoerperschaft}{\hsmakoerperschaftde}%
  \newcommand{\hsmastudiengangname}{Studiengang \hsmastudienganglangde}%
  \newcommand{\hsmastudienganglang}{\hsmastudienganglangde}%
  \newcommand{\hsmatitel}{\hsmatitelde}%
  \newcommand{\hsmatutor}{Betreuer}%
  \newcommand{\hsmafakultaetlang}{\hsmafakultaetlangde}%
  \newcommand{\hsmalistoftables}{Tabellenverzeichnis}%
  \newcommand{\hsmalistoffigures}{Abbildungsverzeichnis}%
  \newcommand{\hsmalistings}{Quellcodeverzeichnis}%
  \newcommand{\hsmaindex}{Index}%
  \newcommand{\hsmaabbreviations}{Abkürzungsverzeichnis}%
  \newcommand{\hsmasnowcardanforderung}{Anforderung}%
  \newcommand{\hsmasnowcardno}{Nr}%
  \newcommand{\hsmasnowcardart}{Art}%
  \newcommand{\hsmasnowcardprio}{Prio}%
  \newcommand{\hsmasnowcardtitel}{Titel}%
  \newcommand{\hsmasnowcardherkunft}{Herkunft}%
  \newcommand{\hsmasnowcardkonflikt}{Konflikte}%
  \newcommand{\hsmasnowcardbeschreibung}{Beschreibung}%
  \newcommand{\hsmasnowcardfitkriterium}{Fit-Kriterium}%
  \newcommand{\hsmasnowcardmaterial}{Weiteres Material}%
  \newcommand{\hsmaqasanforderung}{QAS}%
  \newcommand{\hsmaqasno}{Nr}%
  \newcommand{\hsmaqasart}{Art}%
  \newcommand{\hsmaqasprio}{Prio}%
  \newcommand{\hsmaqastitel}{Titel}%
  \newcommand{\hsmaqasquelle}{Quelle}%
  \newcommand{\hsmaqasstimulus}{Stimulus}%
  \newcommand{\hsmaqasartefakt}{Artefakt}%
  \newcommand{\hsmaqasumgebung}{Umgebung}%
  \newcommand{\hsmaqasantwort}{Antwort}%
  \newcommand{\hsmaqasmass}{Maß für Antwort}%
  \selectlanguage{ngerman}}%
{\newcommand{\hsmatyp}{\hsmatypen}%
  \newcommand{\hsmathesistype}{for the acquisition of the academic degree \hsmagrad}%
  \newcommand{\hsmakoerperschaft}{\hsmakoerperschaften}%
  \newcommand{\hsmastudiengangname}{Course of Studies: \hsmastudienganglang}%
  \newcommand{\hsmastudienganglang}{\hsmastudienganglangen}%
  \newcommand{\hsmatitel}{\hsmatitelen}%
  \newcommand{\hsmatutor}{Tutors}
  \newcommand{\hsmafakultaetlang}{\hsmafakultaetlangen}%
  \newcommand{\hsmalistoftables}{List of Tables}%
  \newcommand{\hsmalistoffigures}{List of Figures}%
  \newcommand{\hsmalistings}{Listings}%
  \newcommand{\hsmaindex}{Index}%
  \newcommand{\hsmaabbreviations}{List of Abbreviations}%
  \newcommand{\hsmasnowcardanforderung}{Requirement}%
  \newcommand{\hsmasnowcardno}{\#}%
  \newcommand{\hsmasnowcardart}{Type}%
  \newcommand{\hsmasnowcardprio}{Prio}%
  \newcommand{\hsmasnowcardtitel}{Title}%
  \newcommand{\hsmasnowcardherkunft}{Origin}%
  \newcommand{\hsmasnowcardkonflikt}{Conflicts}%
  \newcommand{\hsmasnowcardbeschreibung}{Description}%
  \newcommand{\hsmasnowcardfitkriterium}{Fit Criterion}%
  \newcommand{\hsmasnowcardmaterial}{Supporting Material}%
  \newcommand{\hsmaqasanforderung}{QAS}%
  \newcommand{\hsmaqasno}{\#}%
  \newcommand{\hsmaqasart}{Type}%
  \newcommand{\hsmaqasprio}{Prio}%
  \newcommand{\hsmaqastitel}{Title}%
  \newcommand{\hsmaqasquelle}{Source}%
  \newcommand{\hsmaqasstimulus}{Stimulus}%
  \newcommand{\hsmaqasartefakt}{Artifact}%
  \newcommand{\hsmaqasumgebung}{Environment}%
  \newcommand{\hsmaqasantwort}{Response}%
  \newcommand{\hsmaqasmass}{Response Measure}%
  \selectlanguage{english}}%

% Daten in die Standard-Felder von KOMA-Script eintragen
\titlehead{\hsmatyp\ in\  \hsmastudienganglang}
\subject{}
\title{\hsmatitel}
\author{\hsmaauthor}
\date{\small{\hsmadatum}}

% Daten für das fertige PDF-Dokument
\hypersetup{
  pdftitle={\hsmatitel},                           % Titel des Dokuments
  pdfauthor={\hsmaautor},                          % Autor
  pdfsubject={\hsmatyp\ in\ \hsmastudienganglang}, % Thema
  pdfkeywords={\hsmatitel}                         % Schlüsselworte
}

\newlength{\bindekorrektur}
\newlength{\seitenanfang}
\newlength{\seitenbreite}

\setlength{\bindekorrektur}{-46mm}   % Korrektur der horizontalen Position
\setlength{\seitenanfang}{0mm}       % Korrektur der vertikalen Position
\setlength{\seitenbreite}{297mm}     % Breite der Seite

\noindent\includegraphics[width=7cm]{imageuni.pdf}\\

% Titel der Arbeit
\begin{textblock*}{128mm}(45mm,\seitenanfang + 62mm) % 4,5cm vom linken Rand und 6,0cm vom oberen Rand
  \centering\Large\sffamily
  \vspace{4mm} % Kleiner zusätzlicher Abstand oben für bessere Optik
  \textbf{\hsmatitel}
\end{textblock*}%

% Name
\begin{textblock*}{128mm}(45mm,\seitenanfang + 103mm)
  \centering\large\sffamily
  \hsmaautor
\end{textblock*}

% Thesis
\begin{textblock*}{128mm}(45mm,\seitenanfang + 130mm)
  \centering\large\sffamily
  \hsmatyp\\
  \begin{small}\hsmathesistype \end{small}\\
  \vspace{2mm}
  \hsmastudiengangname
\end{textblock*}

% Fakultät
\begin{textblock*}{128mm}(45mm,\seitenanfang + 165mm)
  \centering\large\sffamily
  \hsmafakultaetlang\\
  \vspace{2mm}
  \hsmakoerperschaft
\end{textblock*}

% Datum
\begin{textblock*}{128mm}(45mm,\seitenanfang + 190mm)
  \centering\large
  \textsf{\hsmadatum}
\end{textblock*}

% Betreuer
\begin{textblock*}{128mm}(45mm,\seitenanfang + 240mm)
  \centering\large\sffamily
  \hsmatutor \\
  \vspace{2mm}
  \hsmabetreuer\\
  \vspace{2mm}
  \hsmazweitkorrektor
\end{textblock*}

% Bibliographische Informationen
\null\newpage
\thispagestyle{empty}

\newcommand{\hsmabibde}{\begin{small}\textbf{\hsmaautorbib}: \\ \hsmatitelde \ / \hsmaautor. \ -- \\ \hsmatypde, \hsmaort : \hsmakoerperschaftde, \hsmajahr. \pageref{lastpage} Seiten.\end{small}}

\newcommand{\hsmabiben}{\begin{small}\textbf{\hsmaautorbib}: \\ \hsmatitelen \ / \hsmaautor. \ -- \\ \hsmatypen, \hsmaort : \hsmakoerperschaften, \hsmajahr. \pageref{lastpage} pages. \end{small}}

% Reihenfolge hängt von der Sprache ab
\ifthenelse{\equal{\hsmasprache}{de}}%
{\hsmabibde \\ \vspace{0.5cm} \\ \hsmabiben}
{\hsmabiben \\ \vspace{0.5cm} \\ \hsmabibde}

% Erklärung zur Eigenhändigkeit
\clearpage\setcounter{page}{1}
\thispagestyle{empty}
\textsf{\large\textbf{Erklärung}}

Hiermit erkläre ich, dass ich die vorliegende Arbeit selbstständig verfasst und keine anderen als die angegebenen Quellen und Hilfsmittel benutzt habe.

\ifthenelse{\boolean{hsmapublizieren} \and \not\boolean{hsmasperrvermerk}}%
{
  \vspace{0.5cm}
  Ich bin damit einverstanden, dass meine Arbeit veröffentlicht wird, d.\,h. dass die Arbeit elektronisch gespeichert, in andere Formate konvertiert, auf den Servern der Universität Siegen öffentlich zugänglich gemacht und über das Internet verbreitet werden darf.
}{}%

\vspace{1cm}
\hsmaort, \hsmadatum \\

\vspace{1.2cm}
\hsmaautor

% Sperrvermerk
\ifthenelse{\boolean{hsmasperrvermerk}}%
{%
  \vspace{11cm}
  \color{red}\textsf{\large\textbf{Sperrvermerk}}

  Diese Arbeit basiert auf internen und vertraulichen Daten des Unternehmens \hsmafirma.

  Diese Arbeit darf Dritten, mit Ausnahme der betreuenden Dozenten und befugten Mitglieder des Prüfungsausschusses, ohne ausdrückliche Zustimmung des Unternehmens und des Verfassers nicht zugänglich gemacht werden.

  Eine Vervielfältigung und Veröffentlichung der Arbeit ohne ausdrückliche Genehmigung -- auch in Auszügen -- ist nicht erlaubt.
  \color{black}
}{}

\cleardoublepage

% Abstract
\chapter*{Abstract}

% Reihenfolge hängt von der Sprache ab
\ifthenelse{\equal{\hsmasprache}{de}}%
{
  \subsubsection*{\hsmatitelde}
  \hsmaabstractde
  \begin{otherlanguage}{english}
    \subsubsection*{\hsmatitelen}
    \hsmaabstracten
  \end{otherlanguage}
}
{
  \subsubsection*{\hsmatitelen}
  \hsmaabstracten
  \begin{otherlanguage}{ngerman}
    \subsubsection*{\hsmatitelde}
    \hsmaabstractde
  \end{otherlanguage}
}

% Snowcard
\newcommand{\snowcard}[9]{
  \begin{table}[ht!]
    \caption{\hsmasnowcardanforderung\ #1 -- #4}\label{#1}
    \renewcommand{\arraystretch}{1.2}
    \centering
    \sffamily
    \begin{footnotesize}

      \begin{tabularx}{\linewidth}{sssssb}
        \toprule
        \textbf{\hsmasnowcardno}                           & #1                     & \textbf{\hsmasnowcardart} & #2 & \textbf{\hsmasnowcardprio} & #3 \\
        \midrule
        \multicolumn{2}{l}{\textbf{\hsmasnowcardtitel}}    & \multicolumn{4}{X}{#4}                                                                    \\
        \ifx                                               & #5                     &                                                                  %
        \else
        \multicolumn{2}{l}{\textbf{\hsmasnowcardherkunft}} & \multicolumn{4}{X}{#5}                                                                    \\
        \fi
        \ifx                                               & #6                     &                                                                  %
        \else
        \multicolumn{2}{l}{\textbf{\hsmasnowcardkonflikt}} & \multicolumn{4}{l}{#6}                                                                    \\
        \fi
        \addlinespace
        \multicolumn{6}{l}{\textbf{\hsmasnowcardbeschreibung}}                                                                                         \\
        \multicolumn{6}{F}{#7}                                                                                                                         \\
        \ifx                                               & #8                     &                                                                  %
        \else
        \addlinespace
        \multicolumn{6}{l}{\textbf{\hsmasnowcardfitkriterium}}                                                                                         \\
        \multicolumn{6}{F}{#8}                                                                                                                         \\
        \fi
        \ifx                                               & #9                     &                                                                  %
        \else
        \addlinespace
        \multicolumn{6}{X}{\textbf{\hsmasnowcardmaterial}}                                                                                             \\
        \multicolumn{6}{F}{#9}                                                                                                                         \\
        \fi
        \bottomrule
      \end{tabularx}
    \end{footnotesize}
  \end{table}
}

% Quality Attribute Scenario
\newcommand{\qas}[9]{
  \begin{table}[ht!]
    \caption{\hsmaqasanforderung\ #1 -- #3}\label{#1}
    \renewcommand{\arraystretch}{1.2}
    \centering
    \sffamily
    \begin{footnotesize}

      \begin{tabularx}{\linewidth}{sssssb}
        \toprule
        \textbf{\hsmaqasno}                           & #1                     & \textbf{\hsmaqasart} & QAS & \textbf{\hsmaqasprio} & #2 \\
        \midrule
        \multicolumn{2}{l}{\textbf{\hsmaqastitel}}    & \multicolumn{4}{X}{#3}                                                           \\
        \multicolumn{2}{l}{\textbf{\hsmaqasquelle}}   & \multicolumn{4}{l}{#4}                                                           \\
        \multicolumn{2}{l}{\textbf{\hsmaqasstimulus}} & \multicolumn{4}{X}{#5}                                                           \\
        \multicolumn{2}{l}{\textbf{\hsmaqasartefakt}} & \multicolumn{4}{X}{#6}                                                           \\
        \addlinespace
        \multicolumn{6}{l}{\textbf{\hsmaqasumgebung}}                                                                                    \\
        \multicolumn{6}{F}{#7}                                                                                                           \\
        \addlinespace
        \multicolumn{6}{X}{\textbf{\hsmaqasantwort}}                                                                                     \\
        \multicolumn{6}{F}{#8}                                                                                                           \\
        \addlinespace
        \multicolumn{6}{X}{\textbf{\hsmaqasmass}}                                                                                        \\
        \multicolumn{6}{F}{#9}                                                                                                           \\
        \bottomrule
      \end{tabularx}
    \end{footnotesize}
  \end{table}
}
